\documentclass[12pt,titlepage,draft]{article}
% draft - disables image rendering, shows black boxes at hypernation issues
% Defaults:
%  10pt
%  notitlepage
%  usletter page size on some distros (not mine)
%  singleside (twoside changes the margins for a binding)

% Style:
% http://practicaltypography.com
% https://www.sharelatex.com/learn/Paragraphs_and_new_lines

% Left-aligned text; equal spacing between words, better readability
%  May look "unprofessional", books, articles typically use fully justified
% ragged2e retains hyphenation and attempts to give a more uniform right edge, but maintains equal spacing between words
\usepackage{ragged2e}
\RaggedRight

% Single space after a full stop, modern typographic readability standard
\frenchspacing

% Paragraph spacing with no indent; breaks up huge blocks of text
\usepackage{parskip}

% Increases line spacing (think LaTeX default is okay)
%\usepackage{setspace}
%\onehalfspacing

% Font choices - LaTeX default is computer modern, a bit thin for my liking
% https://www.tug.org/mactex/fonts/LaTeX_Preamble-Font_Choices.html

% Tools
\usepackage{mathtools}
\usepackage{graphicx}
\usepackage{epstopdf}
\usepackage{siunitx}
\usepackage{url}
\usepackage{acronym}
% Some handy commands for referencing (copied from old template)
\newcommand{\figref}[2][\figurename~]{#1\ref{#2}}
\newcommand{\tabref}[2][\tablename~]{#1\ref{#2}}
\newcommand{\secref}[2][Section~]{#1\ref{#2}}
\renewcommand{\vec}[1]{\mathbf{#1}}

% Packages for glossary (List of Symbols)
%\usepackage{glossaries}
%\usepackage[acronym,toc,style=tree]{glossaries}
%\usepackage{longtable}

% Defining symbols
%\newglossaryentry{your-entry}{name=$\theta$, description={Angle of incidence}}

% To use the glossary entry
% \gls{your-entry}

% To print the glossary
% \printglossary[title=List of Symbols]

%\makeglossaries


% Useful LaTeX commands:
% The \clearpage command ends the current page and causes all figures and tables that have so far appeared in the input to be printed.


% General writing tips:
% https://owl.english.purdue.edu/owl/section/1/2/
%   3-5 sentences per paragraph; one idea, contrast, a break, balance
%   15-20 words per sentence; comprehension lowers the longer it gets
%   25-33 syllables; most like typical speech
%   50-75 characters per line, ~65 seems optimal
%   Use plain English; simpler words are better words
%   Explain all technical jargon and abbreviations at first use

% Science writing:
%   Figure text and the abstract should be independent from main text
%   List assumptions after each sub-theory (being clear about problems)
%   Every review is gold dust, people can only read the first time once
%   Send your draft to your references to see if you've cited their work fairly

\begin{document}

\title{Nanocomposite Thermoelectric Materials Theory}
\author{\textbf{Callum Vincent}, Andrew Morris, G.P. Srivastava}
\date{Revision 1 - April 2015}
\maketitle

\tableofcontents

\begin{abstract}
% State the problem
% Why is it interesting?
% What does your solution acheive?
% What follows from your solution?

Thermoelectrics are a promising area of materials research recently revitalised by the introduction of nanocomposites. In this project, we aim to derive a theoretical mechanism, through which new, high efficiency thermoelectric materials can be designed. This will involve a detailed understanding of the fundamental theories of solid-state physics, of which the phonon model plays a critical role. This theoretical project aims to guide expierment, keeping within practical limits and computationally modelling potential designs.
\end{abstract}

\section{Introduction}
\subsection{Motivation}
% Why would you want do this?
Energy and its use defines human society. Throughout human history we have seen an upwards trend of energy consumption and with it we are able to transform our environment and our lives.

Thermoelectric materials have the potential to revolutionise our energy harvesting methods due to their ability to convert heat directly into electricity.

\subsection{Investigation}
% Describe the problem
% Describe the idea
% Make claims about and defend the idea

\subsection{Approach}
% Survey the paper, forward reference the interesting parts, make explicit your contributions to the problem

\section{Background}
% Explain the intuition behind the theory as if on a whiteboard
\subsection{Kinetic Theory}
\paragraph{Assumptions}

\section{Specfics}
% Details of exactly what you've done
% Tell a story, keep it concise, only explain what you need to
\subsection{Thermoelectric Theory}
\paragraph{Assumptions}

\section{Results and Analysis}
% Split into two sections if lengthy
% Explain all elements leading to the results
% Include key graphs, have minimal tables
% What do they show?

\section{Conclusion and Potential Development}
% Summarise results
% What else could be done?

% Link to your online repository
\url{https://github.com/kahlos/thermoelectrics}

\bibliographystyle{IEEEtran}
\begin{thebibliography}{2}
\bibitem{crc-handbook}
G. A. Slack, \emph{CRC Handbook of Thermoelectrics}, 1995, ISBN: 978-0060443849
\bibitem{minnich-review}
A. J. Minnich \emph{et al.}, \emph{Bulk nanostructured thermoelectric
materials: current research and future prospects}, Energy Environ.
Sci. 2, 466-479 (2009), DOI: 10.1039/B822664B
\end{thebibliography}

\end{document}

\newpage
\appendix

\section{Further Questions and Thoughts}

\section{List of Assumptions}

\section{Further Reading}

\section{Tools and Software}

\section{Physical Data}

\section{Program Code}

%abbrivations list%
\acresetall
\ac{PGEC} - Exhibiting the properties of phonon scattering glasses and
electron transmissive crystals
\ac{SSP} - The study of rigid matter (solids)